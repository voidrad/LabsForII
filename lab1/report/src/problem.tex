\CWHeader{Лабораторная работа \textnumero 1}

\CWProblem{
Вы собрали данные и их проанализировали, визуализировали и представили отчет своим партнерам и спонсорам. Они согласились, что ваша задача имеет перспективу и продемонстрировали заинтересованность в вашем проекте. Самое время реализовать прототип! Вы считаете, что нейронные сети переоценены (просто боитесь признаться, что у вас не хватает ресурсов и данных), и считаете что за классическим машинным обучением будущее и потому собираетесь использовать классические модели. Вашим первым предположением является предположение, что данные и все в этом мире имеет линейную зависимость, ведь не зря же в конце каждой нейронной сети есть линейный слой классификации. В качестве первых моделей вы выбрали линейную/логистическую регрессию и SVM. Так как вы очень осторожны и боитесь ошибиться, вы хотите реализовать случай, когда все таки мы не делаем никаких предположений о данных и взяли за основу идею "близкие объекты дают близкий ответ" и идею, что теорема Байеса имеет ранг королевской теоремы. Так как вы не доверяете другим людям, вы хотите реализовать алгоритмы сами с нуля без использования scikit-learn (почти). Вы хотите узнать насколько хорошо ваши модели работают на выбранных вам данных и хотите замерить метрики качества. Ведь вам нужно еще отчитаться спонсорам!

\textbf{Формально говоря вам предстоит сделать следующее:}
\begin{enumerate}
    \item
    Реализовать следующие алгоритмы машинного обучения: Linear/Logistic Regression, SVM, KNN, Naive Bayes в отдельных классах;
    \item
    Данные классы должны наследоваться от BaseEstimator и  ClassifierMixin, иметь методы fit и predict;
    \item
    Вы должны организовать весь процесс предобработки, обучения и тестирования с помощью Pipeline;
    \item
    Вы должны настроить гиперпараметры моделей с помощью кросс валидации, вывести и сохранить эти гиперпараметры в файл, вместе с обученными моделями;
    \item
    Проделать аналогично с коробочными решениями;
    \item
    Для каждой модели получить оценки метрик: Confusion Matrix, Accuracy, Recall, Precision, ROC\_AUC curve;
    \item
    Проанализировать полученные результаты и сделать выводы о применимости моделей;
    \item
    Загрузить полученные гиперпараметры модели и обученные модели в формате pickle  на гит вместе с Jupyter Notebook ваших экспериментов.
\end{enumerate}

}
\pagebreak
